% !TEX TS-program = xelatex
% !TEX encoding = UTF-8 Unicode
% !Mode:: "TeX:UTF-8"

\documentclass{resume}
\usepackage{zh_CN-Adobefonts_external} % Simplified Chinese Support using external fonts (./fonts/zh_CN-Adobe/)
% \usepackage{NotoSansSC_external}
% \usepackage{NotoSerifCJKsc_external}
% \usepackage{zh_CN-Adobefonts_internal} % Simplified Chinese Support using system fonts
\usepackage{linespacing_fix} % disable extra space before next section
\usepackage{cite}

\begin{document}
\pagenumbering{gobble} % suppress displaying page number

\name{殷海洋}

\basicInfo{
  \email{dylanyin@foxmail.com} \textperiodcentered\
  \phone{(+86) 156-9740-5976} \textperiodcentered\
  % \linkedin[billryan8]{https://www.linkedin.com/in/billryan8}
}
 
\section{\faUsers\ 工作/项目经历}

\datedsubsection{\textbf{滚球视频弹幕互动游戏}}{2022 年 11 月 -- 2023 年 8 月}
\role{UE5, C++, Python, Blueprint, Stable Diffusion}

\begin{onehalfspacing}
%   \textbf{游戏玩法}

% \begin{itemize}
%   \item 在视频网站发布视频,观看用户在视频弹幕中竞猜下期中奖的小球。
%   \item 对视频网站中的所有弹幕进行爬取。
%   \item 发布下一期视频,之后得到所有猜对的用户。
%   \item 发布新一期的视频,所有猜对的用户得到标有自己 ID 的小球以及随机的涂鸦,然后参与比赛。
%   \item 冠军用户将得到永久的涂鸦皮肤作为奖励。
% \end{itemize}

        \textbf{技能}

\begin{itemize}
  \item 在 UE 中使用蓝图设计圆形和锥形的轨道,以及对应的道具。
  \item 使用 Python 对视频弹幕爬取,保存为 JSON 文件。
  \item 使用 Stable Diffusion 生成随机涂鸦。
  \item 在 UE 中使用 C++ 对 JSON 文件读取,创建对应小球赋予涂鸦材质,并使用 Text3dComponent 将用户名围绕小球显示。
  \item 在 UE 中使用录制功能对小球进行物理模拟并保存到对应的 Sequence 中。
  \item 在 UE 中使用 Python 将模拟得到的关键帧复制到对应的 Sequence 中,之后渲染得到成片。
\end{itemize}

        \textbf{AI 绘画}

\begin{itemize}
  \item 部署 Stable Diffusion, 下载模型供生成绘画。
  \item 使用 Wildcards 插件保存正反提示词。
  \item 使用 ControlNet 插件对绘画进行控制。
  \item 使用 Lora 对绘画风格进行迁移。
  \item 下载过纹理生成的模型尝试生成纹理,但无法得到很好的法线贴图。
\end{itemize}

\end{onehalfspacing}


\datedsubsection{\textbf{分布式 KV 数据库} }{2022 年 8 月 -- 2022 年 10 月}
\role{Golang, Raft }{}
\begin{onehalfspacing}
  \textbf{主要职责与成果}

\begin{itemize}
  \item 学习 MIT 6.824 公开课,阅读相关论文,基于 Go 开发分布式 KV 数据库。
  \item 采用 Multi-Raft 架构,支持数据分片处理,分片迁移。
  \item Raft 支持 Leader 选举,日志复制,Snapshot 等基本功能。
\end{itemize}


\end{onehalfspacing}



\datedsubsection{\textbf{镜像打包工具}, 硬件可信技术与工程实验室(2012 实验室)}{2021 年 8 月 -- 2022 年 7 月}
\role{Python, Flask, Sqlite}{Python 开发工程师}
\begin{onehalfspacing}
  \textbf{主要职责与成果}

\begin{itemize}
  \item 负责开发与维护镜像打包工具。
        % 该工具主要负责对开发人员编译的二进制文件进行加密、签名、填充,并生成可用于芯片烧写的镜像文件。
  \item 设计并提供用户注册、登录、密钥管理等功能。
        % 成功登录后,用户可对密钥进行申请、删除等操作。
  \item 根据用户上传二进制文件及配置文件完成镜像打包操作。
        % 工具会对上传的文件进行打包,用户也可查看以往的打包任务,并可下载已打包好的文件。
\end{itemize}

        \textbf{技能}

\begin{itemize}
  \item 使用 Flask 和 Bootstrap4,用于提供网页后端与前端设计。
  \item 使用 Cryptography 完成中间加密任务。
  \item 使用 Flask-Migrate 进行数据库表模型的构建与迁移。
  \item 使用 requests 库,完成与加密中心和签名中心的对接。
  \item 使用消息队列框架 RabbitMQ 用于调度和管理打包任务。
\end{itemize}

\end{onehalfspacing}

\datedsubsection{\textbf{ 员工画像}, 深圳国显科技有限公司}{2021 年 2 月 -- 2021 年 8 月}
\role{Python, Flask, PostgreSQL, Pandas}{Python 开发工程师}
\begin{onehalfspacing}
\textbf{主要职责与成果}

\begin{itemize}
\item
通过对工厂员工的产品良率、加班时长、领导评价等多个维度生成用户画像,提供全面的员工绩效与能力评估。
\item
设计和开发了一个基于企业微信的网页应用,为员工提供个性化的信息和权限。
\end{itemize}

\textbf{技能}

\begin{itemize}
\item
使用 Kettle 进行 ETL 操作,定时将相关数据抽取到数据仓库。
\item
利用正则表达式 re 模块,有效地完成了数据的清洗和筛选。
\item
通过 Pandas 库进行数据处理,对用户进行多维度的标签化。
\item
使用 Flask 提供前端 API 接口,实现了前后端的分离。
\item
利用企业微信提供的 OAuth 授权登录方式,实现了员工的权限管理。
\end{itemize}
\end{onehalfspacing}

\datedsubsection{\textbf{ 成品规格书数字化}, 深圳国显科技有限公司}{2020 年 6 月 -- 2021 年 12 月}
\role{Python, Flask, PostgreSQL, Pandas}{Python 开发工程师}
\begin{onehalfspacing}
\begin{itemize}
\item
项目主要职责:对历史上所有成品规格书的 pdf 文档以及 docx 文档解析,将成品的属性提取到数据库中,提供查询历史成品的各个属性的接口。
\item
解析历史上所有的成品规格书的 PDF 和 docx 文档,提取成品的关键属性。
\item
利用 Python-docx 和 Pdfplumber 库对文档进行解析,提取并规范化不同文档中的成品属性数据。
\item
通过 Pandas 和正则表达式模块,对提取到的数据进行清洗和整理,确保数据质量和准确性。
\item
使用 Kettle 工具将清洗后的数据抽取到数据仓库,以供后续使用和查询。
\item
使用 Flask 提供前端 API 接口,允许用户查询历史成品的各个属性。
\end{itemize}
\end{onehalfspacing}


% Reference Test
%\datedsubsection{\textbf{Paper Title\cite{zaharia2012resilient}}}{May. 2015}
%An xxx optimized for xxx\cite{verma2015large}
%\begin{itemize}
%  \item main contribution
%\end{itemize}

\section{\faGraduationCap\  教育背景}
\datedsubsection{\textbf{湖南理工学院}, 湖南}{2016 -- 2020}
\textit{学士}\ 电气工程及其自动化
\section{大学项目}
\datedsubsection{\textbf{图像处理与深度学习}, 大学}{}
\role{Python, Sklearn, Pytorch}{实验室项目}
\begin{onehalfspacing}
高光谱图像分类
\begin{itemize}
  \item 使用 Sklearn 对高光谱图像降维。
  \item 由于高光谱图像数据量少,直接训练很容易造成过拟合,于是使用了一种像素对匹配的方法来增加训练数据量。
  \item 使用 Pytorch 构建卷积神经网络,对高光谱图像进行训练。
\end{itemize}
\end{onehalfspacing}
\role{Python, Sklearn, Pytorch, OpenCV}{ 校赛一等奖}
\begin{onehalfspacing}
人脸检测
\begin{itemize}
  \item 使用利用 Pytorch 构建卷积神经网络,对人脸进行训练。
  \item 使用 OpenCV 将摄像头捕捉的图像处理。
  \item 推理得到人脸的 BBox 框定。
  \item 使用 PyQT 设计 UI 界面。
\end{itemize}
\end{onehalfspacing}

% \section{\faCogs\ IT 技能}
% % increase linespacing [parsep=0.5ex]
% \begin{itemize}[parsep=0.5ex]
%   \item 编程语言: Python, Go
%   \item 工具: (Emacs +Evil), Git, Linux
%   \item 语言: 英语(六级) - 熟练阅读英文文档,习惯用英语在开源社区沟通
% \end{itemize}

% \section{\faHeart\  个人评价}
% \begin{onehalfspacing}
% \begin{itemize}
% \item
% 主力使用 Python,对 Python 的理解深入,擅长编写地道的 Python 代码
% \item
% 熟悉 Python 的高级技巧,对标准库的使用熟练
% \item
% 熟练在 Linux 上开发,熟悉 Linux 中的各种工具
% \item
% 熟练使用 Git 进行版本管理,对 CI/CD 了解
% \item
% 注重工作效率的提升,熟练使用 Emacs 和 Vim 编辑器
% \item
% 坚持良好的编程习惯,注重编写高质量、可读性强的代码,并且习惯为代码编写测试用例
% \item
% 具备良好的英语阅读习惯,习惯通过官方文档学习和解决问题
% \end{itemize}
% \end{onehalfspacing}

% \section{\faHeartO\ 个人兴趣}
% \datedline{\textit{第一名}, xxx 比赛}{2013 年 6 月}
% \datedline{其他奖项}{2015}

% \section{\faInfo\ 其他}
% % increase linespacing [parsep=0.5ex]
% \begin{itemize}[parsep=0.5ex]
%   \item 技术博客: http://blog.yours.me
%   \item GitHub: https://github.com/username
%   \item 语言: 英语(六级) - 熟练阅读英文文档,可以用英语在开源社区沟通
% \end{itemize}

%% Reference
%\newpage
%\bibliographystyle{IEEETran}
%\bibliography{mycite}
\end{document}
