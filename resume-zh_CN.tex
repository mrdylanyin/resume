% !TEX TS-program = xelatex
% !TEX encoding = UTF-8 Unicode
% !Mode:: "TeX:UTF-8"

\documentclass{resume}
\usepackage{zh_CN-Adobefonts_external} % Simplified Chinese Support using external fonts (./fonts/zh_CN-Adobe/)
% \usepackage{NotoSansSC_external}
% \usepackage{NotoSerifCJKsc_external}
% \usepackage{zh_CN-Adobefonts_internal} % Simplified Chinese Support using system fonts
\usepackage{linespacing_fix} % disable extra space before next section
\usepackage{cite}

\begin{document}
\pagenumbering{gobble} % suppress displaying page number

\name{殷海洋}

\basicInfo{
  \email{mrdylanyin@gmail.com} \textperiodcentered\
  \phone{(+86) 156-9740-5976} \textperiodcentered\
  % \linkedin[billryan8]{https://www.linkedin.com/in/billryan8}
}
 
\section{\faGraduationCap\  教育背景}
\datedsubsection{\textbf{湖南理工学院}, 湖南}{2016 -- 2020}
\textit{学士}\ 电气工程及其自动化

\section{\faUsers\ 工作/项目经历}
\datedsubsection{\textbf{图像处理与深度学习}, 大学}{2017 -- 2019}
\role{Pytorch, Sklearn, Python}{实验室项目}
\begin{onehalfspacing}
高光谱图像的分类
\begin{itemize}
  \item 使用 Sklearn 库的 PCA 方法对高光谱图像降维
  \item 利用 numpy pandas 进行数据处理
  \item 使用 Pytorch 构建神经网络,对高光谱图像进行训练
\end{itemize}
\end{onehalfspacing}

\datedsubsection{\textbf{深圳国显科技有限公司}, 深圳}{2020 年 6 月 -- 2021 年 8 月}
\role{Python, Pandas}{数据开发工程师}
员工用户画像
\begin{itemize}
  \item 通过与业务部门调研,理清数据指标体系
  \item 标签设计与开发
  \item 将标签数据聚合保存到数仓
  \item 通过模糊综合评价⽅法为每个标签打分
  \item 设计 RESTful 接⼝
  \item 用户画像可视化
  \item 为业务部门⼈才培养计划提供支撑
\end{itemize}

\datedsubsection{\textbf{华为外包项目}}{2021 年 8 月 -- 2022 年 2 月}
\role{Python, Flask}{Python 开发工程师}
\begin{onehalfspacing}
镜像打包工具(后端开发)
\begin{itemize}
  \item 设计工具整体的框架
  \item 对接密钥管理系统与签名中心,完成镜像的签名和镜像的加密
  \item 提供注册/登录功能,实现密钥的申请
  \item 设计实现镜像配置文件的上传,打包完成后的镜像下载
\end{itemize}
\end{onehalfspacing}

\datedsubsection{\textbf{华为外包项目}}{2022 年 2 月 -- 至今}
\role{Java, CDT}{Java 开发工程师}
\begin{onehalfspacing}
静态代码扫描(工具开发)
\begin{itemize}
  \item 利用 Eclipse CDT 解析 C/Cpp 语法树
  \item 通过 visit 模式,遍历语法树各个节点
  \item 编写逻辑规则,并将检测出来的缺陷保存下来
  \item 集成到 CI,扫描代码
\end{itemize}
\end{onehalfspacing}

% Reference Test
%\datedsubsection{\textbf{Paper Title\cite{zaharia2012resilient}}}{May. 2015}
%An xxx optimized for xxx\cite{verma2015large}
%\begin{itemize}
%  \item main contribution
%\end{itemize}

\section{\faCogs\ IT 技能}
% increase linespacing [parsep=0.5ex]
\begin{itemize}[parsep=0.5ex]
  \item 编程语言: Python, Java, Go, \LaTeX
  \item 工具: (Emacs +Evil), Git
  \item 语言: 英语(六级) - 熟练阅读英文文档,可以用英语在开源社区沟通
\end{itemize}

% \section{\faHeartO\ 个人兴趣}
% \datedline{\textit{第一名}, xxx 比赛}{2013 年 6 月}
% \datedline{其他奖项}{2015}

% \section{\faInfo\ 其他}
% % increase linespacing [parsep=0.5ex]
% \begin{itemize}[parsep=0.5ex]
%   \item 技术博客: http://blog.yours.me
%   \item GitHub: https://github.com/username
%   \item 语言: 英语(六级) - 熟练阅读英文文档,可以用英语在开源社区沟通
% \end{itemize}

%% Reference
%\newpage
%\bibliographystyle{IEEETran}
%\bibliography{mycite}
\end{document}
